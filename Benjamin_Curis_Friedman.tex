\documentclass[11pt]{article}
\usepackage[margin=0.65in]{geometry}
\usepackage{enumitem}
\usepackage[hidelinks]{hyperref}

\setlength{\parskip}{2.5mm} % small spacing between paragraphs
\setlist[itemize]{topsep=0pt, partopsep=0pt, parsep=0pt, itemsep=0.5mm, leftmargin=*} 

\begin{document}
\raggedright 

\begin{center}
{\LARGE Benjamin Curis-Friedman} \\
Driven Software Engineering student with strong problem-solving skills and hands-on experience in Python programming and data analysis. \\
\href{mailto:benjamin.curis-friedman@mail.mcgill.ca}{benjamin.curis-friedman@mail.mcgill.ca} \textbar\ 
\href{https://www.linkedin.com/in/benjaminc-f}{LinkedIn} \textbar\ 
\href{https://github.com/Benjamincf0}{GitHub} \textbar\ 438-346-5109
\end{center}

\vspace{-1mm}
\textbf{EDUCATION} \\
\vspace{1mm}
McGill University, Software Engineering COOP, Montréal, QC \hfill Expected May 2028 \\
Awards: Alma Mater Entrance Scholarship, Class of Engineering '83 Scholarship \\
GPA: 4.0 / 4.0

Stanford University (Online), Machine Learning Specialization \hfill September 2025 \\
Completed independently through Coursera, covering supervised learning, deep learning, unsupervised-learning, and ML engineering practices.\\
Grade: 100\%

\vspace{1mm}
\textbf{RESEARCH \& ENGINEERING EXPERIENCE} \\
\vspace{1mm}
COOP Internship R.A., McGill iSMART AI Lab, Montréal, QC \hfill May 2025 - Present
\vspace{-2mm}
\begin{itemize}
    \item Contributed to a machine learning project predicting human physiological\\ signals from video in collaboration with graduate and undergraduate researchers.
    \item Developed a Python library (Pandas/NumPy) to synchronize and format multimodal\\ datasets, improving processing efficiency.
    \item Built and troubleshot a complete hardware–software recording solution for participant\\ data aggregation.
\end{itemize}


Rocket Team, McGill University, Montréal, QC \hfill Nov 2024 -- May 2025 \\
Flight Computer Software Sub-team: programmed on a custom PCB in C++ using STM32 IDE.

\vspace{1mm}
\textbf{SOFTWARE PROJECTS} \\
\vspace{1mm}
Neural Network Library From Scratch: Handwritten Digit Recognition \hfill July 2025
\vspace{-2mm}
\begin{itemize}
    \item Created a Sequential model library in NumPy similar to Tensorflow achieving $\sim$96\%\\ test accuracy on MNIST handwritten digits dataset.\\
    \item Implemented backpropagation, mini-batch gradient descent, multiple activation functions,\\ cost functions, and customizable layers.\\
    \item Built an interactive real-time digit recognition demo with visualization of predictions using PyGame.
\end{itemize}

Physics Simulator, Personal Project \hfill Sep 2024 \\
Developed a pendulum simulation web app using JavaScript and numerical methods.

ArduCar, Physics for Engineers Class \hfill May 2024 \\
Designed a Bluetooth-controlled electric vehicle with 0.3s latency using Arduino and C++.

WarHacks Hackathon, Concordia University \hfill Jan 2024 \\
Engineered a fully autonomous vehicle to navigate a designated path using Arduino.

Web Chat Application, Personal Project \hfill Feb 2022 \\
Built a dynamic real-time messaging platform with Vue.js and Firebase, integrating\\ authentication and security rules.

\vspace{1mm}
\textbf{SKILLS} \\
\vspace{1mm}
Software: BASH, ZSH, VSCode, NeoVim, Conda, GitHub \\
Programming: Python (NumPy, matplotlib, ), Java, JavaScript, C, C++, HTML, CSS, Arduino, LaTeX \\
Frameworks: Next.js, Firebase, Vue.js, Git, Flutter \\
Languages: Bilingual in French and English \\
Soft Skills: Hardworking, adaptable, collaborative, independent, reliable
\end{document}
